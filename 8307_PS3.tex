\documentclass[12pt]{article}
\usepackage{amsmath, amsthm, amssymb, enumerate, mathrsfs, graphicx, subfig, verbatim, float, pdflscape, rotating, parskip, setspace, tikz, tikz-qtree, url, epstopdf, mathtools, latexsym, flexisym,accents, multirow,diagbox,accents}
\usepackage[framed,numbered,autolinebreaks,useliterate]{mcode}
\usepackage{url}
\usepackage{listings}
\usepackage{pdfpages}
\usepackage{breqn}

\usepackage[margin=1in]{geometry}
\usepackage[round]{natbib}
\addtolength{\parskip}{\baselineskip}
\DeclareMathSizes{12}{13}{7}{7}
\usepackage[bottom]{footmisc}
\newcommand{\ubar}[1]{\underaccent{\bar}{#1}}

\parskip 2pt
\setlength\parindent{0cm}
\begin{document}
\begin{onehalfspace}


\title{Econ 8307\\ Assignment 2 (Spring 2019)}
\author{Jonah Coste, Fred Xu\\George Washington University}
\date{}
\maketitle
\parskip 10pt
\textbf{Question 1}
\begin{enumerate}[1.]
	\item
	$M^* = (1-\delta)M^* + \epsilon$ gives\\ $M^*= \frac{\epsilon}{\delta}$
	\item
	Let $\mu_t(z_i)$ represent the number of firms of type $z_i$ at time t. Then:\\
	$\mu_{t+1}(z_i) = (1-\delta)\sum_{j=1}^N{\mu_t(z_j)f(z_i|z_j)} + \epsilon\psi(z_i)$\\
	Or in terms of matrix algebra:\\
	$\mu_{t+1} = (1-\delta)\mu_tT + \epsilon\Psi$\\
	Where $\mu_t$ is the 1xN measure over firm types, T is the NxN transition matrix (i.e. $t_{ij} = f(z_j|z_i)$), and $\Psi$ is the 1xN probability distribution for new firms (i.e. $\Psi_i = \psi(z_i)$.)
	\item
	$\mu^* = (1-\delta)\mu^*T + \epsilon\Psi$ gives:\\
	$\mu^* = \epsilon\Psi(I-(1-\delta)T)^{-1}$
	
	\item
	\begin{lstlisting}
N = 10;
e = 1;
delta = .1;
beta = 1;
gamma = .1;

psi = zeros(1,N);
T = zeros(N);
I = eye(N);
A= zeros(1,N);

relprob=@(new, old) max(0, beta - gamma*(new-old)^2);

for i = 1:N
    psi(i) = 1/N;
    for j= 1:N
        A(i) = A(i) + relprob(j, i);
    end
end

for i=1:N
    for j = 1:N
        T(i,j) = relprob(j, i) / A(i);
    end
end

mustar = e*psi*inv(I-(1-delta)*T);

mustar'
	\end{lstlisting}
	Prints transposed steady state measure over firm types.
	\begin{lstlisting}
ans =

    0.7503
    0.9625
    1.0950
    1.1011
    1.0912
    1.0912
    1.1011
    1.0950
    0.9625
    0.7503
	\end{lstlisting}
	\end{enumerate}

\textbf{Question 2}
\begin{enumerate}[1.]
    \item 
    Firm's value function:\\
    $V(z_{it}) = \max\limits_{n_{it}}\left[z_{it}n_{it}^\alpha -w_tn_{it} + \beta(1-\lambda) E(V(z_{i,t+1}))\right]$\\
    Or equivalently:\\
    $V(z_{it}) = \max\limits_{n_{it}}\left[z_{it}n_{it}^\alpha -w_tn_{it} + \beta(1-\lambda) \sum\limits_{j=1}^{N}V(z_j)f(z_j|z_{it})\right]$\\
    Transition function of measure of types:\\
    $\mu_{t+1}(z_i) = \sum\limits_{j=1}^{N}\left[(1-\lambda)f(z_i|z_j)\mu_t(z_j)\right] + M_t\psi(z_i)$
    \item
    Suppressing firm index i, firm's value function:\\
    $V(z_t, n_{t-1}) = \max\limits_{n_t}\left[z_tn_t^\alpha -w_tn_t -\tau w_t \max(o, n_{t-1}-n_t) + \beta(1-\lambda) E(V(z_{t+1},n_t)) -\lambda \tau E(w_{t+1}) n_t\right]$\\
    Transition function of measure of types:\\
    $\mu_{t+1}(z', n') = \sum\limits_{z}\sum\limits_{n}\left[(1-\lambda)\textbf{1}(n^*(z, n) = n')f(z'|z)\mu_t(z,n)\right] + M_t\psi(z)\textbf{1}(n'=0)$\\
    Where $n^*(z,n)$ is the argmax from the value function with arguments z and n. $\textbf{1}(a) = 1$ if a is true and $0$ otherwise. 
    \item
    Firm's value function:\\
    $V(z_t) = \max\limits_{n_t, X}\left[z_t n_t^\alpha -w_t n_t + \beta(1-X) E(V(z_{t+1})) - (1-X)k\right]$\\
    $X \in \{0,1\}$\\
    Transition function of measure of types:\\
    $\mu_{t+1}(z_i) = \sum\limits_{j=1}^{N}\left[(1-X^*(z_j))f(z_i|z_j)\mu_t(z_j)\right] + M_t\psi(z_i)$\\
    Where $X^*(z)$ is the argmax from the value function with argument z.
\end{enumerate}

\textbf{Question 3}
\begin{enumerate}[1.]
    \item 
    Finding $n^*(z)$ where $z \in \{1,2,...,N\}$:
    \begin{lstlisting}
N = 5;
p = .8;
w = 1;
alpha = .7;
beta = .95;

Z = zeros(1,N);
nstar = zeros(N,1);
for i =1:N
    Z(i) = i;
    nstar(i) = (w/(Z(i)*alpha))^(1/(alpha-1));
end

nstar
    \end{lstlisting}
    Prints $n^*$:
    \begin{lstlisting}
nstar =

    0.3046
    3.0697
   11.8594
   30.9405
   65.0969
    \end{lstlisting}
    Not sure what "aggregate labor input" means in this context.
    \item
    I think we need $\psi$ to do the rest of this?
    \item
    \item
    \item
\end{enumerate}

\end{onehalfspace}
\end{document}
