\documentclass[12pt]{article}
\usepackage{amsmath, amsthm, amssymb, enumerate, mathrsfs, graphicx, subfig, verbatim, float, pdflscape, rotating, parskip, setspace, tikz, tikz-qtree, url, epstopdf, mathtools, latexsym, flexisym,accents, multirow,diagbox,accents}
\usepackage[framed,numbered,autolinebreaks,useliterate]{mcode}
\usepackage{url}
\usepackage{listings}
\usepackage{pdfpages}
\usepackage{breqn}

\usepackage[margin=1in]{geometry}
\usepackage[round]{natbib}
\addtolength{\parskip}{\baselineskip}
\DeclareMathSizes{12}{13}{7}{7}
\usepackage[bottom]{footmisc}
\newcommand{\ubar}[1]{\underaccent{\bar}{#1}}

\parskip 2pt
\setlength\parindent{0cm}
\begin{document}
\begin{onehalfspace}


\title{Econ 8307\\ Assignment 2 (Spring 2019)}
\author{Jonah Coste, Fred Xu\\George Washington University}
\date{}
\maketitle
\parskip 10pt
\textbf{Question 1}
\begin{enumerate}[1.]
	\item
	$M^* = (1-\delta)M^* + \epsilon$ gives\\ $M^*= \frac{\epsilon}{\delta}$
	\item
	Let $\mu_t(z_i)$ represent the number of firms of type $z_i$ at time t. Then:\\
	$\mu_{t+1}(z_i) = (1-\delta)\sum_{j=1}^N{\mu_t(z_j)f(z_i|z_j)} + \epsilon\psi(z_i)$\\
	Or in terms of matrix algebra:\\
	$\mu_{t+1} = (1-\delta)\mu_tT + \epsilon\Psi$\\
	Where $\mu_t$ is the 1xN measure over firm types, T is the NxN transition matrix (i.e. $t_{ij} = f(z_j|z_i)$), and $\Psi$ is the 1xN probability distribution for new firms (i.e. $\Psi_i = \psi(z_i)$.)
	\item
	$\mu^* = (1-\delta)\mu^*T + \epsilon\Psi$ gives:\\
	$\mu^* = \epsilon\Psi(I-(1-\delta)T)^{-1}$
	
	\item
	\begin{lstlisting}
N = 10;
e = 1;
delta = .1;
beta = 1;
gamma = .1;

psi = zeros(1,N);
T = zeros(N);
I = eye(N);
A= zeros(1,N);

relprob=@(new, old) max(0, beta - gamma*(new-old)^2);

for i = 1:N
    psi(i) = 1/N;
    for j= 1:N
        A(i) = A(i) + relprob(j, i);
    end
end

for i=1:N
    for j = 1:N
        T(i,j) = relprob(j, i) / A(i);
    end
end

mustar = e*psi*inv(I-(1-delta)*T);

mustar'
	\end{lstlisting}
	Prints transposed steady state measure over firm types.
	\begin{lstlisting}
ans =

    0.7503
    0.9625
    1.0950
    1.1011
    1.0912
    1.0912
    1.1011
    1.0950
    0.9625
    0.7503
	\end{lstlisting}
	\end{enumerate}

\textbf{Question 2}
\begin{enumerate}[1.]
    \item 
    Firm's value function:\\
    $V(z_{it}) = \max\limits_{n_{it}}\left[z_{it}n_{it}^\alpha -w_tn_{it} + \beta(1-\lambda) E(V(z_{i,t+1}))\right]$\\
    Or equivalently:\\
    $V(z_{it}) = \max\limits_{n_{it}}\left[z_{it}n_{it}^\alpha -w_tn_{it} + \beta(1-\lambda) \sum\limits_{j=1}^{N}V(z_j)f(z_j|z_{it})\right]$\\
    Transition function of measure of types:\\
    $\mu_{t+1}(z_i) = \sum\limits_{j=1}^{N}\left[(1-\lambda)f(z_i|z_j)\mu_t(z_j)\right] + M_t\psi(z_i)$
    \item
    Suppressing firm index i, firm's value function:\\
    $V(z_t, n_{t-1}) = \max\limits_{n_t}\left[z_tn_t^\alpha -w_tn_t -\tau w_t \max(o, n_{t-1}-n_t) + \beta(1-\lambda) E(V(z_{t+1},n_t)) -\lambda \tau E(w_{t+1}) n_t\right]$\\
    Transition function of measure of types:\\
    $\mu_{t+1}(z', n') = \sum\limits_{z}\sum\limits_{n}\left[(1-\lambda)\textbf{1}(n^*(z, n) = n')f(z'|z)\mu_t(z,n)\right] + M_t\psi(z)\textbf{1}(n'=0)$\\
    Where $n^*(z,n)$ is the argmax from the value function with arguments z and n. $\textbf{1}(a) = 1$ if a is true and $0$ otherwise. 
    \item
    Firm's value function:\\
    $V(z_t) = \max\limits_{n_t, X}\left[z_t n_t^\alpha -w_t n_t + \beta(1-X) E(V(z_{t+1})) - (1-X)k\right]$\\
    $X \in \{0,1\}$\\
    Transition function of measure of types:\\
    $\mu_{t+1}(z_i) = \sum\limits_{j=1}^{N}\left[(1-X^*(z_j))f(z_i|z_j)\mu_t(z_j)\right] + M_t\psi(z_i)$\\
    Where $X^*(z)$ is the argmax from the value function with argument z.
\end{enumerate}

\textbf{Question 3}
\begin{enumerate}[1.]
    \item 
    Finding $n^*(z)$ where $z \in \{1,2,...,N\}$:
    \begin{lstlisting}
N = 5;
p = .8;
w = 1;
alpha = .7;
beta = .95;

Z = zeros(1,N);
nstar = zeros(N,1);
for i =1:N
    Z(i) = i;
    nstar(i) = (w/(Z(i)*alpha))^(1/(alpha-1));
end

nstar
    \end{lstlisting}
    Prints $n^*$:
    \begin{lstlisting}
nstar =

    0.3046
    3.0697
   11.8594
   30.9405
   65.0969
    \end{lstlisting}
    The aggregate labor input is:
    $\sum\limits_{z}\mu_t(z) n^*(z)$
    Or in matrix notation: $\mu_t n^*$ where $mu_t$ is the 1xN measure over types and $n^*$ is the the Nx1 vector of optimal employment over types.
    \item
    We will assume that that the distribution for new firms $\psi$ is a uniform distribution.
    \begin{enumerate}
        \item Without $n_{t-1}$ as a state variable. $\mu^*=\mu^*(z)$
        \begin{lstlisting}
T = zeros(N);
T=T+(1-p)/(N-1)+eye(N)*(p-(1-p)/(N-1));
I=eye(N);
psi= zeros(1,N);
for i =1:N
    psi(i) = 1/N;
end

mustar = psi*inv(I-(1-lambda)*T);
mustar1 = mustar/sum(mustar);
mustar1'
    \end{lstlisting}
Prints steady state distribution of types.
\begin{lstlisting}
ans =

    0.2000
    0.2000
    0.2000
    0.2000
    0.2000
\end{lstlisting}
        \item With $n_{t-1}$ as a state variable. $\mu^*=\mu^*(z_t, n_{t-1})$
        \begin{lstlisting}
N = 5;
p = .8;
alpha = .7;
beta = .95;
lambda = .1;
E = 1;
wguess = 1;
tau = .0;

Z = zeros(1,N);

for i =1:N
    Z(i) = i;
end

T = zeros(N);
T=T+(1-p)/(N-1)+eye(N)*(p-(1-p)/(N-1));
I=eye(N);
psi= zeros(1,N);
for i =1:N
    psi(i) = 1/N;
end

%Find equilibrium wage
step = 1;
gridsize = 100;
gridmax = 100;

while step > .001
% Find Value function given wguess

nvalues = zeros(gridsize+1,1);
for i = 1:gridsize+1
    nvalues(i)= gridmax*(i-1)/gridsize; 
end

nstar = zeros(N, gridsize+1);
Vold = zeros(N, gridsize+1);
temp = zeros(gridsize+1,1);
loss = 1;

while loss > .01
Vnew = zeros(N, gridsize+1);
for i = 1:N
    for j = 1:gridsize+1
        temp = zeros(gridsize+1,1);
        for k = 1:gridsize+1
            temp(k) = Z(i)*nvalues(k)^alpha - wguess*nvalues(k) - tau*wguess*max(0,nvalues(j)-nvalues(k)) + beta*(1-lambda)*T(i,:)*Vold(:,k) - beta*lambda*tau*wguess*nvalues(k);
        end
        [M, I]= max(temp);
        nstar(i,j) = nvalues(I);
        Vnew(i,j) = Z(i)*nstar(i,j)^alpha - wguess*nstar(i,j) - tau*wguess*max(0,nvalues(j)-nstar(i,j)) + beta*(1-lambda)*T(i,:)*Vold(:,I) - beta*lambda*tau*wguess*nstar(i,j);
    end
end
loss = sum(abs(Vnew-Vold), 'all');
Vold = Vnew;
end
V = Vnew;
gridmax = round(max(nstar,[],'all')*1.1,2,'significant');
Ecalc = beta*psi*V(:,1);
if Ecalc-E >0
    wguess = wguess + step;
else
    wguess = wguess - step;
    step = step/2;
end
end

%iterate to find steady state measure
muold = ones(N, gridsize+1);
loss2=1;

while loss2 > .01
munew = zeros(N, gridsize+1);
for i = 1:N
    for j = 1:gridsize+1
        for ii = 1:N
            for jj = 1:gridsize+1
                if nstar(ii,jj) == nvalues(j)
                    munew(i,j) = munew(i,j) + (1-lambda)*T(i,ii)*muold(ii,jj);
                end
            end
        end
        if nvalues(j) == 0
            munew(i,j) = munew(i,j) + psi(i);
        end
    end
end
loss2 = sum(abs(munew-muold), 'all');
muold = munew;
end

mm = munew/sum(munew,'all');
mu = sum(mm')
\end{lstlisting}
Prints steady state distribution of types.
\begin{lstlisting}
mu =

    0.2000    0.2000    0.2000    0.2000    0.2000
\end{lstlisting}
Computing proportion of jobs destroyed.
\begin{lstlisting}
jobs = 0;
dest = 0;
for i= 1:N
    for j = 1:gridsize+1
        jobs = jobs + mm(i,j)*nstar(i,j);
        dest = dest + lambda*mm(i,j)*nstar(i,j);
        for k = 1:N
                kI = find(nvalues==nstar(i,j));
                dest = dest + (1-lambda)*mm(i,j)*T(i,k) * max(0,(nstar(i,j)-nstar(k,kI)));
            end
        end
end
propdest = dest/jobs
\end{lstlisting}
\begin{lstlisting}
propdest =

    0.2267
\end{lstlisting}
    \end{enumerate}
    
    
    \item Uses same code as 2b with tau parameter changed. List only results for equilibrium wage and job destruction for each value of tau. Results rounded because of computational restrictions.
\begin{lstlisting}
tau =  0, wage = 5.88, job destruction = .227
tau = .5, wage = 5.33, job destruction = .203
tau =  1, wage = 4.92, job destruction = .181
\end{lstlisting}
The cost of firing lowers wages. Job destruction varies a similar amount as wages.
    \item
\begin{lstlisting}
omega = 1;
c = wguess/omega;
y = 0;
emp = 0;
for i = 1:N
    for j = 1:gridsize+1
        y = y + Z(i)*nstar(i,j)^alpha * munew(i,j);
        emp = emp + nstar(i,j)*munew(i,j);
    end
end
Mstar = c/(y-E);
emp = emp*Mstar
\end{lstlisting}
Prints steady state employment. Showing results for each value of tau.
\begin{lstlisting}
tau =  0, employment = .870
tau = .5, employment = .795
tau =  1, employment = .745
\end{lstlisting}
    \item
    Since $\omega=1$ consumption = wages which is 5.88 in the base (tau = 0) case. Steady state one period utility is $C - \omega n = 5.01$ in the base case. To maintain this utility, consumption would need to be increased by $\frac{5.01+.795}{5.33}-1 = 8.9\%$ if tau = .5 and by $\frac{5.01+.745}{4.92}-1 = 17\%$ if tau = 1.
\end{enumerate}

\end{onehalfspace}
\end{document}
